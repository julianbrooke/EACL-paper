

%\usepackage[latin1]{inputenc} % for PCs

\title{Formulaic Sequence Annotation Guidelines, Croatian supplement}
\author{Jan \v{S}najder}
\date{September 19, 2016}

\documentclass[12pt]{article}

\usepackage[utf8]{inputenc}
\usepackage[croatian]{babel}
%\usepackage[T1]{fontenc}

\begin{document}
\maketitle
 

Upute za označavanje ze engleski jezik u vrijede i za hrvatski jezik. Primjeri formulaičnih izraza u
hrvatskome (popis nije iscrpan):

\begin{itemize}
\item Imeničke fraze:
\begin{itemize}
\item ``električna vuča'', ``kontrola kvalitete'', ``investicijski fond'', ``drugi svjetski rat'', ``Katica
za sve'', ``mali od palube''
\item Tu su osobito česte kombinacije pridjev+imenica ili imenica+imenica-u-genitivu, a moguće su
naravno i složenije kombinacije
\end{itemize}
\item Imena poznatih entiteta (osoba, mjesta, organizacija) – sintaktički gledano, to su također imeničke
fraze:
\begin{itemize}
\item ``Zagrebačka županija'', ``Fakultet elektrotehnike i računarstva'', ``Justin Bieber'', ``Marks
and Spencer'', ``Balkanska ruta''
\end{itemize}

\item Povratni glagoli:
\begin{itemize}
\item Pravi povratni (riječ ``se'' korištena je kao zamjenica i može se zamijeniti sa ``sebe''): ``predati
se'', ``upustiti se''
\item Uzajamno povratni: ``grliti se'', ``pozdraviti se'',
\item Nepravi povratni (nema prave povratnosti radnje na subjekt, ``se'' je  čestica i ne može se
zamijeniti sa ``sebe''): ``glasati se'', ``veseliti se'', ``čuditi se'', ``smijati se''
\end{itemize}
\item Glagolske fraze:
\begin{itemize}
\item ``pridobiti pažnju'', ``biti u žiži'', ``dati sve od sebe'', ``baciti * u vjetar'', ``može se govoriti
o *'', ``dolaziti na naplatu'', ``stići na naplatu''
\end{itemize}
\item Pridjevske fraze:
\begin{itemize}
\item ``koliko toliko'', ``s time povezan'', ``bez presedana''
\end{itemize}
\item Priložne oznake:
\begin{itemize}
\item ``bezbroj puta'', ``svakim danom sve *'', ``malo puno''
\end{itemize}
\item Prijedložne oznake:
\begin{itemize}
\item ``u bližoj okolici'', ``u nepoznatom smjeru'', ``u ovom trenutku''
\end{itemize}
\item Diskursni konektori:
\begin{itemize}
\item ``sve dok je'', ``shodno tome'', ``drugim riječima'', ``na primjer''
\end{itemize}
\item Klišeizirane fraze:
\begin{itemize}
\item ``plan i program'', ``može * biti sram'', ``mama ti mogu biti'', ``čiča miča gotova je priča'',
``bolje išta nego ništa'', ``nepoznati netko'', ``možda su u šumi'', ``nema govora o *''
\end{itemize}
\end{itemize}

U odnosu na engleski, kod hrvatskoga jezika nailazimo na neke specifičnosti, ponajviše zbog razlika u
morfologiji (visok stupanj fleksije i derivacije) i sintaksi (relativno slobodan poredak riječi u rečenici). S
druge strane, u hrvatskome ne postoje neki fenomeni koji postoje u engleskome, npr., članovi i frazalni
glagoli.


\section*{Morfologija}

Za razliku od engleskog, hrvatski je visoko flektivan jezik, što znači da se jedna te ista fraza može
pojavljivati u različitim flektivnim oblicima, npr., ``jedan te isti'', ``jedna te ista'', ``jednog te istog'' (primijetite, međutim, da je u nekim slučajevima fraza okamenjena i ne dopušta nikakvu morfološku
varijaciju, npr., ``Katicama za sve'' ili ``u bližim okolicama'' nisu formulaički nizovi). Pri označavanju, fraza će
biti prikazana u obliku u kojem se ona najčešće nalazi u korpusu; taj oblik ne mora nužno biti nautknički
oblik kakav biste našli u rječniku. Npr., formulaički niz ``dvostruka igra'' pojavljuje se u korpusu najčešće u
kosom obliku  ``dvostruku igru''  (jer se fraza najčešće koristi kao objekt u akuzativu), pa će taj oblik biti
prikazan, međutim oba oblika se odnose na isti formulaički izraz i trebaju se tretirati ravnopravno. Dodatno
će biti prikazan i potpuno lematizirani oblik izraza (``dvostruki igra''), koji može, ali i ne mora, biti
gramatički ispravan oblik izraza u hrvatskom jeziku. Pri označavanju treba dakle zanemariti sve ove razlike
i označiti formulaičnost niza neovisno o tome u kojem je morfološkom obliku prikazan. Osobito treba obratiti
pažnju na supletivne oblike, posebice glagola biti (sam/si/je/smo/ste/su) i slično. Npr.,  ``je u žiži'' je
formulaični niz, kao što je to i niz ``su u žiži'' i ``smo u žiži'', što su sve flektivne varijante niza ``biti u žiži''.
Budući da se provodi lematizacija, ovi se nizovi tretiraju kao istovjetni, što znači da se u popisu ne bi smjele
zasebno pojaviti varijante ovog niza.


Dodatna razlika hrvatskoga u odnosu na engleski jezik u pogledu morfologije jest ta što hrvatski koristi
derivacijsku morfologiju kako bi kodirao neke semantičke značajke poput spola osobe koja obavlja
zanimanje (tzv.~mocijski parovi) ili glagolskog aspekta (svršenost i nesvršenost glagola), npr., ``kuhar'' –
``kuharica'', ``dobitnik'' – ``dobitnica'', ``stići'' – ``stizati'', ``komentirati'' – ``prokomentirati''.  Ove razlike neće biti
uklonjene lematizacijom, što znači da se, primjerice, fraze ``goli kuhar'' i ``gola kuharica'' tretiraju kao
različite, te formulaičnost svake od njih treba ocijeniti zasebno.


Druga značajna razlika hrvatskoga u odnosu na engleski jezik jest ta što hrvatski dopušta relativno
slobodan poredak riječi u rečenici. Posljedica toga jest da se u nekim formulaičnim nizovima može mijenjati
poredak riječi u rečenici (doduše, ta je pojava rijetka, jer formulaičnost obično diktira poredak riječi) ili, što
se događa znatno češće, da se neke riječi mogu umetati u formulaične izraze, da bi se zadovoljila
sintaktička pravila hrvatskog ili iz stilskih razloga. 


\section*{Sintaksa}
Za engleski je već bilo objašenjeno da je moguće umetanje riječi u ``rupe'' (engl.~\emph{gaps}) formulaičnog niza. U
hrvatskom se relativno često pojavljuju rupe koje mogu popuniti modifikatori. Npr. ``stvoriti * atmosferu'' ili
``kazna * zatvora''. Te rupe su često opcionalne, u smislu da je i niz bez rupe također formulaičan.

Također su moguće rupe koje popunjavaju objekti. Npr., u formulaičnom nizu ``baciti * u vjetar'', rupu će
tipično popuniti objekt, izrečen kao imenička fraza ili zamjenica (``baciti novce u vjetar'', ``bacio ih u vjetar'',
``bacio sve svoje novce u vjetar''). Kao i kod engleskog, u ovakvim slučajevima fraze ``baciti novce u vjetar''
ili ``bacio ih u vjetar'' nisu formulaične, jer riječi ``novce'' i ``ih'' varijabilan dio niza te ih se može tretirati kao
argumente formulaičnog niza. Shodno tome, poopćeni niz ``baciti * u vjetar'' jest formulaičan niz (s
rupom).

Međutim, za razliku od engleskog, kod hrvatskog će rupe često nastajati zbog slobodnijeg poretka riječi u
rečenici, i takve će rupe uvijek biti opcionalne. Npr., formulaični niz ``zadnji trenutak'' može se pojaviti u
verziji s rupom, ``zadnji * trenutak'', gdje rupu može popuniti glagolska kopula (``je'', ``smo'', ``ste'', ``su'', …)
ili neki složeniji niz (npr., ``ga je'', ``smo ju'', \dots). Oba ova niza tretiramo kao formulaična.


Također vrijedi i obrat: formulaični nizovi koji se sastoje od rupe, npr. ``baciti * u vjetar'', mogu se pojaviti
bez rupe. Npr., objekt se može naći prije ili poslije formulaičnog niza, kao u ``Baca u vjetar dva milijuna
kuna.'' ili ``Ta odluka baca u vjetar sve što je momčad napravila.'' Također, opet je moguće da rupu popune
riječi koje se tu moraju naći zbog sintakse, npr. glagolska kopula ``biti'' za tvorbu perfekta ili futura: ``bacila
je u vjetar'', ``bacit će u vjetar''. Također su moguće i kombinacije, gdje rupu zajedno popunjavaju i
kopula i objekt, npr. ``bacila je sve svoje novce u vjetar''.

U nastavku slijedi sistematizacija fenomena uzrokovanih slobodnim poretkom riječi u rečenici:

\begin{itemize}

\item Transpozicija riječi u formulaičnom nizu:
\begin{itemize}
\item Transpozicija zamjeničke enklitike (povratne zamjenice ``se''):
\begin{itemize}
\item ``veseliti se'' $\Rightarrow$ ``se veseli''
\item  ``dati si vremena'' $\Rightarrow$ ``si dati vremena''
\end{itemize}
\item Transpozicija glagolske enklitike (kopule):
\begin{itemize}
\item ``lako je tako'' $\Rightarrow$ ``je lako tako'', ``tako je lako''
\end{itemize}
\item U ovakvim slučajevima, sve transponirane varijante formulaičnog niza tretiramo kao
istovrijedne. Drugim riječima, dok god su u nizu iste riječi, može ih se različito poredati i opet
dobiti formulaični niz (naravno, samo za one poretke riječi koje su u jeziku dopuštene; npr.
``dati vremena si'' nije formulaičan niz).
\end{itemize}

\item Umetanje riječi u formulaičan niz:
\begin{itemize}
\item Umetanje glagolske enklitike (kopule):
\begin{itemize}
\item ``zadnji trenutak'' $\Rightarrow$ ``zadnji je trenutak'' $\Rightarrow$ ``zadnji * trenutak''
\item ``hrvatska banka'' $\Rightarrow$ ``hrvatska je banka'' $\Rightarrow$ ``hrvatska * banka''
\item ``zatvoriti poglavlje'' $\Rightarrow$ ``zatvorio je poglavlje'' $\Rightarrow$ ``zatvoriti * poglavlje''
\end{itemize} 
\item Umetanje zamjeničke enklitike:
\begin{itemize}
\item ``zadnji trenutak'' $\Rightarrow$ ``zadnji se trenutak'' $\Rightarrow$ ``zadnji * trenutak''
\end{itemize}
\item Kombinacije:
\begin{itemize}
\item ``zadnji trenutak'' $\Rightarrow$ ``zadnji su ga trenutak'' $\Rightarrow$ ``zadnji * trenutak''
\end{itemize}
\item Kod umetanja riječi, formulaičan niz s umetnutom riječi ne tretira se kao formulaičan, jer je
umetnuta riječ donekle varijabilna (u svim gornjim primjerima na mjesto rupe moguće umetnuti
jednu riječ ili kombinaciju više riječi). Drugim riječima, umetnuti dio treba zamijeniti rupom, a
da bi se niz tretirao kao formulaičan.
\end{itemize}
\end{itemize}

Na primjer, ``zatvoriti poglavlje'' je formulaičan niz (po kriteriju semantičke neprozirnosti). Zbog slobodnog
poretka riječi u rečenici, niz ``zatvoriti * poglavlje'' također tretiramo kao formulaičan. Rupu može popuniti
kopula (``zatvorila je poglavlje'') ili neka složenija kombinacija (``zatvorili su jučer nakon dugih pregovora
poglavlje''), međutim nizovi s popunjenom rupom nisu formulaični jer su previše specifični. Niz ``je zatvorila
poglavlje'' također nije formulaičan, jer kopula ``je'' nije dio niza.

\end{document}

